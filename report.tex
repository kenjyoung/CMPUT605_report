\documentclass{article}
\usepackage{fullpage}
\usepackage{lastpage}
\usepackage{fancyhdr}
\usepackage{amssymb}
\usepackage{amsmath}
\usepackage{mathtools}
\usepackage{enumerate}
\usepackage{xspace}
\usepackage{txfonts}
\usepackage{biblatex}
\begin{document}
\begin{center}
{\Large \textbf{CMPUT 605: Exploring Methods for Guiding Exploration in Learning to Play Hex}}
\end{center}
\section*{Introduction}
The aim of this project was initially to investigate a particular method for guiding exploration in the board-game Hex in order to facilitate learning of approximate win probabilities for each available move in a state using a deep convolutional neural network. This method was based on the idea that the goal of exploration should be to follow lines of play that maximally reduce the Bernoulli variance in the state win probability estimate (i.e. $P_w(1-P_w)$) . This initial goal was found to be seemingly infeasible (at least in the particular form which was investigated here). Therefore other methods of improving learning in the game of Hex were also investigated, including exploration guided by pseudo-counts and using information generated by an existing hex solver to guide the learning and exploration.

For brevity, throughout we will use the notation $x\leftsquigarrow y$ to denote a gradient descent update which modifies the parameters of $x$ in order to reduce $(x-y)^2$.

\section*{Exploration by Estimated Variance Reduction}
The first idea tried for guiding exploration was based on learning probability estimates for each move based on the assumption that each move being a win was an independent Bernoulli event with some initially unknown probability. Expanding from this the idea was to also learn an estimated reduction in state win probability variance that would result by taking each move, playing out the rest of the game to the end, and performing a full backup of all move probabilities discovered along the way.

let $Q(S,A)$ be our win probability (or action value) estimate for each action $A$ in a given state $S$. Ordinary ones-step Q-learning (for 2 player minimax games) performs updates of the form:
$$Q(S_t,A_t)\leftsquigarrow 1-\max_A Q(S_{t+1},A)$$
In order to iteratively improve the estimated action values. Notice that if the game were stochastic and $Q(S_t,A_t)$ was equal to the true win probability, this update would be at a fixed point hence in case of stochastic games it makes some sense to say that $Q(S_t,A_t)$ is our estimate of the true winning probability of each move. Hex on the other hand is deterministic so each move has a true win probability of either 1 or 0, hence it is perhaps less clear what intermediate values of $Q(S_t,A_t)$ imply, however it can be roughly thought of as our degree of belief about the state being a win.

Given this interpretation we suggest an alternative update rule that makes this idea somewhat more explicit, and which will allow us to formulate the concept of state value variance reduction resulting from each particular move. Our new update rule assumes each move independently has a true value of either a win or a loss and therefore to do one step backups we look at the joint probability over the next state's moves as follows:
$$Q(S_t,A_t)\leftsquigarrow \prod_A 1-Q(S_{t+1},A)$$
i.e. the probability that the last move was a win should be the joint probability that every move in the resulting state is a loss for the opponent.

With this new formulation in mind we can get a state win probability as the probability that any move in the state is a win (i.e. not every move is a loss):
$$V(S_t)=1-\prod_A 1-Q(S_t,A)$$
Furthermore we can formulate the variance in this state win probability and attempt to choose moves which seek to reduce this variance using a tree backup of a maximally variance reducing rollout to the end of the game. The state value variance is given by:
$$\prod_A Q(S_t,A)-\prod_A Q(S_t,A)^2$$
Or notion of optimal exploration is as follows: we wish to follow some sequence of actions to the end of the game such that if we perform full backups of the form $Q(S_t,A_t)\leftarrow \prod_A 1-Q(S_{t+1},A)$ the reduction in the variance of out current state value estimate $\prod_A Q(S_t,A)-\prod_A Q(S_t,A)^2$ will be maximal. To simplify the problem and allow us to learn an estimate of expected variance reduction via one step backups we make the assumption that the expected change in $Q(S_t,A)$ resulting from taking action A and then following this procedure is 0. This is quite hand-wavy and perhaps ill-motivated, however it makes some sense in light of the fact that if we were to attempt to learn this change using the same model we use to learn $Q(S_t,A)$ itself, we would expect it to quickly converge to 0 simply because the model is already attempting to learn $Q(S_t,A)$ via one step backups. If we \textit{could} learn the expectation of this change was something other than 0 we could simply add this expectation to the current value estimate in order to generate a more accurate estimate without actually performing the lookahead.

With this assumption then our task remains to estimate the change in $Q(S_t,A)^2$ along a sequence of actions aiming to maximize this change. More concretely, call this quantity:
$$\Delta \sigma(S_t,A_t) = E[\max_{A_{t+1}}E[\max_{A_{t+2}}...(Q(S_t,A_t)^2-\prod_A Q_{new}(S_t,A_t)^2)...]]$$
where $Q_{new}(S_t,A)$ represents the new value $Q(S_t,A)$ after backing up along the followed action sequence $(A_t,A_{t+1},...)$. Now note that:
\begin{align*}
Q_{new}(S_t,A) &=Q(S_t,A) \text{ for } A\neq A_1\\
Q_{new}(S_t,A_t) &=\prod_A 1-Q_{new}(S_{t+1},A)\\
Q_{new}(S_t,A_t)^2&=\prod_A 1-2Q_{new}(S_{t+1},A)+Q_{new}(S_{t+1},A)^2\\
\end{align*}
\begin{align*}
\Delta \sigma(S_t,A_t) &= E[\max_{A_{t+1}}E[\max_{A_{t+2}}...( Q(S_t,A_t)^2- Q_{new}(S_t,A_t)^2)...]]\\
&=E[\max_{A_{t+1}}E[\max_{A_{t+2}}...((Q(S_t,A_t)^2-(\prod_{A} (1-Q(S_{t+1},A))^2+(\prod_{A} (1-Q(S_{t+1},A))^2-Q_{new}(S_t,A_t)^2)...]]\\
\intertext{using our assumption that $E[Q_{new}(S_t,A_t)]=Q(S_t,A_t)$ and the expansion of $Q_{new}(S_t,A_t)^2$ noted above... }
&=E[\max_{A_{t+1}}E[\max_{A_{t+2}}...(Q(S_t,A_t)^2-\prod_{A} (1-Q(S_{t+1},A))^2+\prod_{A\neq{A_{t+1}}} (1-Q(S_{t+1},A))^2(Q(S_{t+1},A_{t+1})^2-Q_{new}(S_{t+1},A_{t+1})^2)...]]\\
&=E[\max_{A_{t+1}}E[\max_{A_{t+2}}...(Q(S_t,A_t)^2-\prod_{A} (1-Q(S_{t+1},A))^2+\prod_{A\neq{A_{t+1}}} (1-Q(S_{t+1},A))^2\Delta \sigma(S_{t+1},A_{t+1}))...]]\\
&=E[(Q(S_t,A_t)^2-\prod_{A} (1-Q(S_{t+1},A))^2+\max_{A_{t+1}}\prod_{A\neq{A_{t+1}}} (1-Q(S_{t+1},A))^2\Delta \sigma(S_{t+1},A_{t+1}))]
\end{align*}
We have derived a recursive formula which we can now use to specify the following update rule for an exploration score network:
$$Q_{exp}(S_t,A_t)\leftsquigarrow Q(S_t,A_t)^2-\prod_{A'} (1-Q(S_{t+1},A'))^2+\max_A\prod_{A'\neq A} (1-Q(S_{t+1},A'))^2Q_{eAxp}(S_{t+1},A)$$
And we can define an "exploration greedy action selection" by noting that our goal in choosing actions should be to reduce the variance in the current state value estimate by as much as possible:
$$A_t= \text{argmax}_A\prod_{A'\neq A}(1-Q(S_t,A))^2Q_{exp}(S_t,A)$$
During training the idea was to select actions using an epsilon greedy version of this metric while simultaneously updating both the $Q$ and $Q_{exp}$ networks which would share some number of lower layers. 

This was attempted, however the joint distribution based update rule for $Q$ was not found to be feasible for learning on reasonable sized boards and hence there was no real opportunity to even test the $Q_{exp}$ network, which relies on $Q$.  Testing on 5x5 showed that the backup method could learn to some degree as shown in figure \cite{5x5_scores}. However even on 5x5 it was still qualitatively outperformed by ordinary Q-Learning and performance degraded very rapidly when scaling to larger boards. For these reasons we did not bother to proceed with more rigorous testing of this method at this time. The reason for the failure appears to be simply that the method gets quickly lost in noise, and on larger boards more moves means there is far more noise to get lost in. The backup rule is essentially very pessimistic because if there are a large number of moves with even a small winning chance in the next state it will conclude that a loss is likely because the opponent has so many possible wins. This issue could potentially be remedied with more sophisticated statistical techniques and may be worth examining further in the future.
%insert figure 1 here 


\section*{Pseudocount Exploration}


\section*{Integrating a Hex solver with the Learning Process}

\section*{Results}

\section*{Conclusion}

\section*{Future Work}

\begin{thebibliography}{9}
\bibitem{psuedocounts}
Bellemare, Marc, et al. "Unifying count-based exploration and intrinsic motivation." \textit{Advances in Neural Information Processing Systems}. 2016.
\end{thebibliography}



\end{document}